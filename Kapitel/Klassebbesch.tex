\chapter{Klassenbeschreibung}
\subsection{WAMSPlugin}
Main Class of the WAMS Plugin for Airflow. Ties together the other Classes in the plugin and implements the AirflowPlugin interface.
%Wrapper Klasse die die Hauptlogik des Plugins enthält und das AirflowPlugin interface implementiert.

\subsection{flask\_appbuiler.BaseView}
Klasse die für das Anzeigen der einzelnen Komponenten des Plugins veranwortlich ist.

\section{CodeEditor}
\subsection{CodeEditorController}
Class that connects the Code editor plugin with the WAMS plugin and connection to Git and Tree.

\subsection{CodeEditorView}

\subsection{ResultView}
Class that is responsible for the apperance of the Results part of the workflows. 

\subsection{MetricsView}
Klasse die Darstellung von der Metadaten Ansicht in der Anwendung zuständig ist. 
Die Klasse erweitert flask\_appbuiler.BaseView.

\subsubsection{MetricsFetcher}
Class that is responsible for the fetching of the Metadata of a specific workflow from the Airflow Database.

\subsection{Json View}
Klasse die für das Darstellen von Json Datein zuständig ist.

\section{Operators}
\subsection{PingPongOperator}
%Operator class used for dynamically created tasks like the theory guided maschine learning modell.
This operator receives two python callables. It executes both of them alternating as often as specified in the repetition parameter of its \textit{\_\_init\_\_()} method. The results of the callables can be used as input parameters for each repetition to allow increasing accuracy of the model. The first callable can be started with paramters also passed in the \textit{\_\_init\_\_()} method.


\subsection{TriggerTillSatOperator}
This operator receives two python callables. In opposition to the \textit{PingPongOperator} the number of repetition is not set via the \textit{\_\_init\_\_()} method but can change dynamically. The first callable triggers the second one every time the second one has finished its work until it is satisfied. The threshold value deciding if the second callable should be run again can also be passed in the \textit{\_\_init\_\_()} method. The first callable receives the return values of the second one after every run of the second one.


\section{Ergebnispaket}
\subsection{ResultsExplorerView}
Klasse die für das Anzeigen der Ergebnisse in dem Dateiexplorerer zuständig ist.
Die Klasse erweitert flask\_appbuiler.BaseView.


\section{Klassen von außerhalb} %Airflow, code editor etc
Klassen die nicht von uns Geschrieben wurden/werden aber trotzdem für den Aufbau relevant sind da diese erweitert werden oder referenziert werden.

\subsection{AirflowPlugin interface}
Das Interface von Airflow das implementiert werden muss damit ein Plugin für Airflow nutzbar ist. Wird von WAMSPlugin implementiert

\subsection{BaseOperator}
Klasse von Airflow die einen BaseOperator von Airflow definiert. 

\subsection{Airflow API}
API von Airflow

\subsection{List Dags (resource)}
Resource die eine liste von allen DAGs ist.

\subsection{List Dag runs (resource)}
Resource die eine liste von allen DAG runs ist. 

\subsection{flask.Blueprint}
Klasse von flask mit der Templates gehandhabt werden können.

\subsection{code\_editor\_plugin\_blueprint: Blueprint}
Implementiert die flask.Blueprint Klasse/das Interface. 



\subsubsection{templates}

\subsubsection{static}

\subsubsection{utils.py}
%Python Library Klasse 
\subsubsection{commony.py}

\subsubsection{tree.py}

