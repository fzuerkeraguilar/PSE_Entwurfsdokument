\chapter{Einleitung}
WAMS wird als Plugin in die Rahmenarchitektur von Airflow eingebunden.
Dies ermöglicht folgende Funktionalitäten über die Funktionen von Airflow hinaus: 
\begin{itemize}
    \item Funktionen von Airflow können direkt übernommen werden.
    \item Die aktuelle Airflow Version kann verwendet werden.
    \item Unkomplizierte Inbetriebnahme von WAMS als Airflow Plugin
\end{itemize}
%
%Dies ermöglicht das Apache Airflow unabhängig von WAMS aktualisiert werden kann.
%Dies verbessert die Erweiterbarkeit von WAMS durch die Installation von Airflow Plugins.
%Dies ermöglicht das WAMS erweiterbar ist durch die Installation von Airflow Plugins.
%Dies ermöglicht eine Verwendung von der aktuellen Airflow Version selbst bei updates.

\section{Paketstruktur}

WAMS ist in mehrere Pakete unterteilt. In diesen sind Klassen mit ähnlichen Funktionen zusammengefasst.


Da WAMS eine Erweiterung von Airflow ist werden unter anderem folgende Funktionen von Airflow übernommen:
\begin{itemize}
    \item Weboberfläche
    \item Web Server
    \item Workflowspeicherung
    \item Parametrierung von Workflows
    \item Datenbank für Metadaten
    \item Ablaufplanung von Workflows
    \item Ausführung von Workflows
\end{itemize}
WAMS bindet an unterschiedlichen Stellen des Modells von Airflow an.
Daher wird für WAMS keine Model-View-Controller Struktur verwendet.
%Da die Erweiterungen von WAMS an vielen Stellen des Modells von Airflow anbinden, wird für WAMS keine Model-View-Controller Struktur verwendet.
Stattdessen wird WAMS in Pakete aufgeteilt, welche jeweils einer Funktionalität von WAMS entsprechen.

%Dabei liegt der Fokus auf einer Aufteilung in unterschiedliche Funktionen. 
%Das führt dazu, dass WAMS modular aufgebaut ist. 
%So können einzelne Funktionen von WAMS auch für andere ähnliche Plugins verwendet werden.
%Im Vergleich zur Aufteilung der Funktionen in mehrere Pakete ist dies durch die gewählte Paketstruktur einfach möglich.
%Es müssen lediglich die Pakete in ein anderes Plugin übernommen werden. 
Die Hauptaufgaben von WAMS liegen im Bereitstellen mehrere Funktionen in der Weboberfläche.
Folgende Pakete mit Funktionen stellt WAMS bereit:
\begin{itemize}
    \item Code Editor
    \item Result View
    \item Metadata Explorer
    \item Operators
\end{itemize}
Diese Pakete werden in den nächsten Abschnitten einzeln behandelt.
Weiter greift WAMS auf mehrere Pakete innerhalb von Apatche Airflow (\textit{airflow}) zu, insbesondere auf das Paket \textit{airflow.plugin\_manager} und auf das Paket \textit{airflow.api}. Das Paket \textit{airflow.plugin\_manager} enthält dabei die Schnittstelle, die die Einbindung von WAMS als Plugin ermöglicht.
Das Paket \textit{airflow.api} enthält die Anwendungsprogrammierschnittstelle von Apatche Airflow.

\subsection{Code-Editor}
Der Code-Editor dient dazu den Python-Quellcode von Workflows in der Weboberfläche ändern zu können.
Hierbei wird das Airflow Plugin \gls{airflow-code-editor} von WAMS verwendet.

%Der hier kommt nicht als Paket im Klassendiagramm vor.
%\subsection{File-Explorer}
%Der File-Explorer dient dazu die Ordnerstruktur der Workflows und ihrer bei der Ausführung benötigten oder erzeugten Daten einzusehen und verändern zu können. 
%So realisieren die in ihm enthaltenen Klassen etwa auch die Funktion, dass Developer Konfigurationsdateien und andere Textdateien hochladen können, wenn ein Workflow dies zur Ausführung benötigt.
\subsection{Result View}
Das Result View Paket erweitert die Weboberfläche von Airflow mit einem Tab zum Einsehen der Ergebnisse der Ausführungen von Workflows.


\subsection{Metadata Explorer}
Das Metadata Explorer Paket erweitert die Weboberfläche von Airflow mit einem Tab zum Einsehen der Metadaten einer Ausführung von Workflows. Die Metadaten werrden über die Airflow API ermittelt.

\subsection{Operators}
Das Operators Paket bündelt die in WAMS enthaltene Erweiterung der bereits in Apatche Airflow existierenden Operators. 
In ihm sind Operatoren enthalten, die zu einer einfachen Anwendung von TGDS Workflows gedacht sind.
Außerdem ist ein Operator enthalten, der das Abspeichern von Ergebnissen und Zwischenergebnissen stark vereinfacht.