\makeglossaries
\newglossaryentry{airflow-code-editor}{
name = airflow-code-editor,
description = {Airflow-code-editor ist ein open-source Plugin (Apache-2.0 Lizenz) für Airflow, welches das programmierbasierte Editieren und Erstellen von DAGs in der Weboberfläche von Airflow ermöglicht. \nolinkurl{https://github.com/andreax79/airflow-code-editor}}
}

\newglossaryentry{AppBuilder}{ %Hier sollte noch mehr hin wahrscheinlich
name = \textit{AppBuilder},
description = Flask-AppBuilder ist ein Development-Framework welches auf Flask basiert. Es wird von
Airflow ab Version 2.0.0 für die Darstellung der Weboberfläche verwendet.
\nolinkurl{https://github.com/dpgaspar/Flask-AppBuilder}}

\newglossaryentry{Op}{ %Hier sollte noch mehr hin wahrscheinlich und was ist hier die Formale Definition hab mich dran versucht
name = Op,
description = Der Server Operator ist ein Nutzer welcher Schreibzugriff auf die grundlegende Konfiguration der Anwendung hat. Er ist somit in der Lage die Konfiguration der Serverstruktur der
Anwendung zu ändern.}

\newglossaryentry{Codemirror}{
name = Codemirror,
description = {Ist ein Editor zum Bearbeiten von Code in Webbrowsern welcher auf JavaScript basiert. 
Er ist open-source und wird steht unter der MIT-Lizenz. \nolinkurl{https://github.com/codemirror/CodeMirror}}
%\nolinkurl{https://github.com/codemirror/CodeMirror}
}

\newglossaryentry{Apache Airflow}{
name = Apache Airflow,
description ={Open-source Workflow Management Plattform für Workflows}
}